\subsection{L'EJB stockOption}
Un Ejb EjbStockOption est utilisé pour cette étude. Cet Ejb a pour charge de fournir la liste des stockOptions et de permettre leur création et suppression. Il possède également une méthode changesQuotes() qui fait varier les actions en cours. 

Cet Ejb est un singleton (grâce a l'annotation \verb|@Singleton|), tout les clients qui récupéreront l'Ejb accéderont à la même instance. On est assuré d'avoir un référentiel commun pour tout les clients.

\subsection{L'EJB SONotifier}
Cet ejb a pour charge d'envoyer sur un topic (nommé \verb|StockOptionTopic|) les informations de mise à jour des StockOptions (ajout, modification, suppression). Il est appelé par l'ejb stockOption.

\subsection{OpenEjb}
Il est nécessaire d'avoir un conteneur d'ejb pour utiliser des ejb. Pour cela, nous avons choisis OpenEjb. Ce conteneur présente l'avantage d'être relativement léger (quelques Mo), simple à utiliser et encapsulable dans une application. De plus, il charge automatiquement les ejb qui sont présents dans son classpath. Il est ainsi facile de lancer le serveur pour des tests unitaires et, à terme, pour une application stand-alone. 

Un autre avantage de OpenEjb est qu'il contient un serveur JMS (Java Messaging Service) : OpenMQ.
