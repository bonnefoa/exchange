\subsection{Architecture}
L'interface est une interface Swing. Elle doit permettre à l'utilisateur d'interagir avec le système. Lors de l'élaboration de cette interface, nous avons hésité sur le design pattern à choisir. Dans l'optique de faciliter les tests, le pattern PassiveView (\url{http://martinfowler.com/eaaDev/PassiveScreen.html}) a été utilisé. Nous allons étudier ce pattern.
\subsubsection{Passive View}
Ce pattern est un derivé du pattern MVC avec quelques nuances. La première différence est le fait que la Vue et le Model ne se connaissent pas. Ils sont totalement découplés l'un de l'autre. Le contrôleur se charge de faire la liaison avec les deux. La vue est donc basique. Les changements de la vue sont effectués par le contrôleur.

L'avantage de ce pattern est le fait que la vue est tellement basique qu'il n'y a pas besoin de la tester. Les tests peuvent se limiter aux interactions modèle/contrôleur.

Le désavantage est la surcharge de travail au niveau du contrôleur. Cela peut surcharger la classe et la rendre moins lisible. Cependant, pour les petites applications comme celle-là, cela suffit amplement.
\subsection{Partie cliente}
L'interface permet à l'utilisateur de se connecter au serveur ou de se connecter à la partie administrateur.
\subsubsection{Inscription aux stockOptions}
La partie cliente possède la liste des stockOptions sur la partie gauche. Elle permet de s'inscrire et se désinscrire à la surveillance des stockOptions. Pour cela, il faut simplement sélectionner les stockOptions voulues et cliquer sur ``Subscribe'' ou ``Unsubscribe''.
\subsubsection{Variations des quotes}
La partie droite affiche les messages de variations de quotes des stockOptions pour lesquelles l'utilisateur s'est inscrit.
\subsection{Partie administrateur}
La partie administrateur permet d'ajouter et de supprimer des stockOptions. Le mot de passe est ``adminadmin''.
\subsubsection{Création de stockOptions}
Pour créer un stockOption, il faut rentrer le nom de la compagnie, le nom du titre ainsi que le montant de la quote.

La quote doit être un chiffre positif. Tous les champs sont obligatoires.
\subsubsection{Suppression de stockOptions}
Pour supprimer une stockOption, il suffit de sélectionner les stockOptions à supprimer et cliquer sur ``supprimer''.
\subsection{Tests}
Les tests unitaires ont été réalisés avec fest (Fixture Easy Software Testing). Fest offre une API pour tester simplement les interfaces en simulant un utilisateur utilisant l'interface.
